%{{{ Preamble
\documentclass[titlepage, oneside, a4paper, 11pt]{article}

%%  Packages
%%%%%%%%%%%%%%%%%%%%%%%%%%%%%%%%%%%%%%%%%%%%%%%%

%%  Encoding
%%%%%%%%%%%%%%%%%%%%%%%%%%%%%%%%
\usepackage[T1]{fontenc}
\usepackage[utf8]{inputenc}
% \usepackage[swedish]{babel}

%%  Science tools
%%%%%%%%%%%%%%%%%%%%%%%%%%%%%%%%
\usepackage{amsmath}
% \usepackage[version=3]{mhchem}
\usepackage{siunitx}
\usepackage{subdepth}

%%  Graphics
%%%%%%%%%%%%%%%%%%%%%%%%%%%%%%%%
% \usepackage{graphicx}
% \usepackage{tikz}
% \usepackage{float}

%%  Layout
%%%%%%%%%%%%%%%%%%%%%%%%%%%%%%%%
\usepackage[top=3cm, bottom=3cm, left=2.5cm, right=2.5cm]{geometry}

% toc section dots
% \usepackage{tocloft}
% \renewcommand{\cftsecleader}{\cftdotfill{\cftdotsep}}

%%  Misc.
%%%%%%%%%%%%%%%%%%%%%%%%%%%%%%%%
% \usepackage{multirow}
\usepackage{hyperref}

%%  Tweaks
%%%%%%%%%%%%%%%%%%%%%%%%%%%%%%%%%%%%%%%%%%%%%%%%

% Less cramped tables.
% \renewcommand{\arraystretch}{1.4} %or 1.5
% \renewcommand{\tabcolsep}{0.2cm}

% Newline instead of indent for new paragraphs.
% \setlength{\parindent}{0pt}
% \setlength{\parskip}{\baselineskip}


%%     Title, author etc.
%%%%%%%%%%%%%%%%%%%%%%%%%%%%%%%%%%%%%%%%%%%%%%%%
\def\title{Interactive two-dimensional n-body simulation on
  gravitational force} 
\def\course{Matematik breddning: Mathematical modeling} 
\def\author{Viktor Qvarfordt} 
\def\examiner{Charles Sacilotto} 
\def\school{Internationella Kunskapsgymnasiet}

\usepackage{fancyhdr}
\lhead{\textsc{Matematik Breddning: Mathematical Modeling}}
\rhead{\thepage}
\cfoot{}
\rfoot{\textsl{\author}}
\pagestyle{fancy}

\renewcommand{\footrulewidth}{0.4pt}  

\begin{document}
%}}}

%{{{ Titlepage
\begin{titlepage}
  \thispagestyle{empty}
  \null
  \vskip -2em%
  \noindent \large{\textbf{\school \hfill \today}}
  \vskip 10em
  \begin{center}
    \vskip 2em
    \huge{\textbf{\title}} \\
    \vskip 2em
    \Large{\course} \\
    \vskip 10em
    \begin{large}
      \begin{tabular}{ll}
        \textbf{Author} & \author \\
        \textbf{Examiner} & \examiner \rule{0pt}{1.5em}
      \end{tabular}
    \end{large}
    \vfill
  \end{center}
\end{titlepage}
%}}}

%%%%%%%%%%%%%%%%%%%%%%%%%%%%%%%%%%%%%%%%%%%%%%%%%%%%%%%%%%%%%%%%

\section{Background \& Problem Formulation}


\section{Description of program}

This program is a basic $n$-body simulator, it simulates the
gravitational force exerted between multiple bodies in two dimensions.

New `planets' are produced by pressing and holding down the left mouse
button at the location where the planet is to be created. Initial
velocity is controlled by dragging the mouse while holding down the
left mouse button, this produces the planets velocity vector. The
planet is created when the button is released. All planets are cleared
by pressing space.

Velocity is measured in distance (pixels) per second, ie the length of
the initial user-created velocity vector is the distance the planet
will travel per second. A trail is drawn behind all planets, to help
visualize the planetary orbits.

Both velocity and acceleration is visually shown as vectors
originating in each planets center, drawn in blue and red
respectively. (They are also visually scaled down by a factor of 4 and
8.)

If two planets start overlapping each other a perfectly inelastic
collision will be simulated; the larger body (the one with the largest
mass) will consume the smaller -- momentum is conserved.

Further information on the program and technical details are found in
the attached \texttt{README} file or by running the program with the
help flag \verb|--help|.



\section{Description of calculations}

The calculations are made in steps, that is, the planets are not
progressing through their orbits continuously. At each step the
current state of the system (mainly position, velocity and
acceleration) is identified and used to calculate the next state.

By shortening the interval between each step a more accurate
approximation to the motion of the system is acquired.

The calculations are based on Newtons's law of gravitation
\begin{equation}
  \label{grav}
  F = G\frac{m_1m_2}{d^2}
\end{equation}
and the equations of motion.

At each step in the calculations, for each planet, the resulting sum
of acceleration due to the gravitational force in relation to all
other planets is calculated
\begin{equation}
  \label{sum}
  a_{tot} = \sum_{i=0}^{n}G\frac{m_{i}}{d_{i}^2}
\end{equation}
where $m_{i}$ denotes the mass of the planet by which gravitational
force is exerted, and $d_{i}$ is the distance to the $i$:th
planet, and $n$ is the number of planets. Now that the acceleration is known, the relation
\begin{equation}
  \label{x2nd}
  \frac{d^2x}{dt^2} = a
\end{equation}
can be used to get the new position, $x$, as needed to progress the
system one step further. However, because what is
being solved is a $n$-body system, this cannot be solved exactly for
all planets, known as the ``n-body
problem''.\cite{wiki:n-body_problem} Hence the calculations must be
performed with numerical methods, this is why the simulation is made
up of multiple small steps as earlier mentioned. 

Solving second-order differential equations is a very inefficient
task compared to methods existing for first order DEs. Thus the
integration done when solving for $x$ in \eqref{x2nd} is split up into
two firs-order DEs, 
\begin{align}
  \frac{dv}{dt} &= a \\
  \frac{dx}{dt} &= v \rule{0pt}{23pt}
\end{align}
which will be solved for $v$ and $x$ sequentially. 

In its simplest form an approximated solution can be found by using
the Euler method, defined as
\begin{equation}
  y_{n+1} = y_n + hy'_n
\end{equation}
This means that for each calculation step the following will be
calculated, after the acceleration has been calculated
\begin{align}
  v_{n+1} &= v_{n} + a_{n} \cdot \Delta t \\
  \label{pos}
  x_{n+1} &= x_{n} + v_{n} \cdot \Delta t
\end{align}

If all planets have the internal variables \texttt{position, velocity,
  acceleration} and \texttt{mass}, and a list \texttt{planets}
containing all active planet objects is created, an implementation of
the calculation loop in pseudocode could read:
\begin{verbatim}
for elements in planets as planet1:
    planet1.acceleration = 0
    for elements in planets as planet2:
        if planet1 is not planet2:
            planet1.acceleration += G*planet2.mass/(planet2.pos-planet1.pos)^2
    planet1.velocity += planet1.acceleration * dt
    planet1.position += planet1.velocity * dt
\end{verbatim}
Note that this example is incomplete, as for example the acceleration
becomes a scalar value. Handling this is too technical for the purpose
of demonstrating the method used, see the attached source code for
details.

\section{Optimization}
While the shown method is simple and easy to implement, it suffers
from low accuracy as the truncation error per step is very large. To
yield acceptable results the step-size has to be far smaller than the
computational power on a standard desktop computer allows. 

Some improvements to the method can be done by using the mean value for
the velocity:
\begin{equation}
  v_{mean} = \frac{v(t) + v(t+\Delta t)}{2} = v(t) +
  \frac{a(t)}{2} \Delta t
\end{equation}
using this in the expression for the position \eqref{pos} gives a
better approximation to the position
\begin{align}
  x_{n+1} &= x_{n} + v_{mean} \Delta t \\
         &= x_{n} + v_{n} \Delta t +
  \frac{a_n}{2}(\Delta t)^2
\end{align}
Although this modification greatly increases the accuracy of the
approximation, too small time-steps are required. 

Several methods have been developed for numerically solving
differential equations with higher precision. One of the more stable
methods is the Runge-Kutta method.

Runge-Kutta is essentially a collection of higher order approximations
of solutions to ordinary DEs. (Technically the Euler method can be
seen as the first-order Runge-Kutta method.) The fourth-order
Runge-Kutta (RK4) is one of the most commonly applied, because of its
high convergence relative the step-size.

The concept that the method is based around is that instead of simply
making the step-size smaller, the direction of the step will be
carefully chosen, so that a higher convergence is achieved. This is
the main flaw in the Euler method -- the method only evaluates
derivatives at the beginning of the interval, at $x_n$. The RK4 method
evaluates the derivatives at four points and calculates a weighted
mean value of the derivatives at each of these point, this mean value
is then used as the final derivative, the direction in which the next
step will be performed.

Letting $y' = f(x, y)$, the standard fourth-order Runge-Kutta method takes the generalized form:
\begin{align*}
k1 &= f\left(x_n, y_n\right) \\
k2 &= f\left(x_n + \frac{h}{2}, y_n + \frac{hk_1}{2}\right) \\
k3 &= f\left(x_n + \frac{h}{2}, y_n + \frac{hk_2}{2}\right) \\
k4 &= f\left(x_n + h, y_n + hk_3\right) \\
y_{n+1} &= y_n + \frac{h}{6}(k1 + k2 + k3 + k4)
\end{align*}


\newpage
\begin{thebibliography}{99}
\bibitem{wiki:n-body_simulation}
  \url{http://en.wikipedia.org/wiki/N-body_simulation}, 2011-05-01
\bibitem{wiki:n-body_problem}
  \url{http://en.wikipedia.org/wiki/N-body_problem}, 2011-05-01
\bibitem{rk4}
  \url{http://farside.ph.utexas.edu/teaching/329/lectures/node35.html},
  2011-05-01
\bibitem{euler}
  \url{http://en.wikipedia.org/wiki/Euler_method}, 2011-05-01
\bibitem{verlet}
  \url{http://www.lonesock.net/article/verlet.html}, 2011-05-01
\bibitem{wiki:rk}
  \url{http://en.wikipedia.org/wiki/Runge-Kutta_methods}, 2011-05-01
\end{thebibliography}

\end{document}
